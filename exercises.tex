\documentclass[10pt]{article}
\usepackage{amsmath}
\usepackage{amssymb}
\usepackage{amsthm}
\usepackage[lastexercise]{exercise}

\renewcommand{\QuestionNB}{(\alph{Question})\ }

\begin{document}

\section{Induction Proofs}
\begin{ExerciseList}
    \Exercise
    Show that for any $n\in\mathbb N_1$
    $$\frac{1}{1 \cdot 3}+\frac{1}{3 \cdot 5}+\ldots+\frac{1}{(2n-1)(2n+1)}=\frac{n}{2n+1}$$

    \Exercise
    Consider the sequence $(a_n)$ defined by
    \[
    \begin{cases}
        a_1=6,\\
        a_{n+1}  = \frac{a_n}{3}(1-e^{-a_n^2}), & \mbox{if $n\ge 1$}\\
    \end{cases}
    \]
    \Question{Show that $a_n\in (0,6]$ for any $n\in\mathbb N_1$}
    \Question{Show that the sequence is decreasing}
    \Question{Prove that the sequence is convergent and determine it's limit}

    \Exercise
    Consider a sequence $(b_n)$ defined by
    \[
    \begin{cases}
        b_1=\frac{1}{2},\\
        b_{n+1}  = b_n(b_n - 1) + 1, & \mbox{if $n\ge 1$}\\
    \end{cases}
    \]
    \Question{Show that $b_n\in (0,1)$ for any $n\ge 1$}
    \Question{Show that the sequence is increasing}
    \Question{Prove that the sequence is convergent and determine it's limit}

    \Exercise
    Consider the sequence $(u_n)$ defined by
    \[
    \begin{cases}
        u_1=1,\\
        u_{n+1}  = 1 + \frac{u_n}{2}, & \mbox{if $n\ge 1$}\\
    \end{cases}
    \]
    \Question{Show that $u_n \le 2$ for any $n\in\mathbb N_1$}
    \Question{Show that the sequence is increasing}
    \Question{Prove that the sequence is convergent and determine it's limit}

    \Exercise
    Let $\alpha\in [0,1]$, consider the sequence $(b_n)$ given by
    \[
    \begin{cases}
        b_1=\alpha,\\
        b_{n+1}  = b_n - b_n^4, & \mbox{if $n\ge 1$}\\
    \end{cases}
    \]
    \Question{Show that $b_n$ is monotonically decreasing}
    \Question{Show that $b_n\in [0,1]\forall n\in\mathbb N_1$}
    \Question{Prove that $b_n$ is convergent and calculate $\lim b_n$}

    \Exercise
    Consider the sequence $(a_n)$ defined by
    \[
    \begin{cases}
        a_1=1,\\
        a_{n}  = \frac{3a_{n-1}}{n}, & \mbox{if $n\ge 2$}\\
    \end{cases}
    \]
    Show that
    $$a_n=\frac{3^{n-1}}{n!}\quad\quad\forall n\ge 1$$
\end{ExerciseList}

\section{Limits I}
\begin{ExerciseList}
    \Exercise Calculate or show that it does not exist in $\overline{\mathbb R}$
    \Question $\lim \frac{(n+1)!-n!}{n!(n+2)}$
    \Question $\lim (-1)^n \frac{10^n}{n!}$
    \Question $\lim \frac{5n!+5n}{n^n + 2}$
    \Question $\lim \sqrt{\frac{e^n+2}{n!}}$
    \Question $\lim \frac{(-1)^n n}{n!+4}$
    \Question $\lim \frac{\sqrt[3]{n+4}}{\sqrt[3]{n}+4}$
    \Question $\lim \sqrt[n]{\frac{n+2^n}{2+5^n}}$
    \Question $\lim \frac{3n^4-2n}{(n^2 + 3)(1+5n^2)}$
    \Question $\lim (\frac{2}{3}+\cos(3n\pi))^n$
    \Question $\lim \frac{3+\cos(e^{-n})}{n+\sqrt{n!}}$
    \Question $\lim \frac{(-1)^n n }{n! + 5}$
    \Question $\lim \frac{\cos{(n\pi)}}{\sin{(1/n\pi)}+1}$
    \Question $\lim \frac{e^{3n}+1}{2^n+n^2}$
    \Question $\lim \frac{\sqrt{n} + n^3}{(n+\sqrt{n})(n^2 + n^{3/2})}$
    \Question $\lim (\cos{(\frac{\pi}{4}+n\pi)}+1)^n$
    \Question $\lim \frac{(3n)!}{n!(2n)!}$
    \Question $\lim \frac{\arccos(1/n)}{\cos(\pi/n)}$
    \Question $\lim \frac{2n! + 3^n}{n^{50} + n!}$
    \Question $\lim \sqrt[n]{\frac{\arctan n}{1+e^n}}$
\end{ExerciseList}

\section{Limits II}
\begin{ExerciseList}
    \Exercise Calculate or show that it does not exist in $\overline{\mathbb R}$
    \Question $\lim_{x\to 0}\frac{e^x - 1}{x-e^{3x} + 1}$
    \Question $\lim_{x\to +\infty}(3x^2+1)^{1/x}$
\end{ExerciseList}

\section{Functions}
\begin{ExerciseList}
    \Exercise Consider the function $f\colon \mathbb R \to \mathbb R$
    $$f(x) = \begin{cases} -x e^x, & \mbox{if $x<0$} \\ \alpha\arctan(x^2-2x), &\mbox{if $x\ge 0$} \\ \end{cases}$$
    Where $\alpha$ is a real constant.
    \Question Calculate $\lim_{x\to -\infty}f(x)$ and $\lim_{x\to +\infty}f(x)$
    \Question Justify whether $f$ is continuous
    \Question Let $f'_+(0)=-2\alpha$, determine $\alpha$ such that $f$ is differentiable at $x=0$. Justify that $f$ is differentiable in $\mathbb R$ and calculate it's derivative
    \Question Determine the local extremes and the monotonous intervals of $f$ (with $\alpha = 1/2$)
    \Question Indicate the co-domain of $f$ (with $\alpha=1/2$)

    \Exercise The function $h\colon \mathbb R \to \mathbb R$
    $$h(x)=\begin{cases}x^2, & \mbox{if $x\in\mathbb Q$} \\ 0, & \mbox{if $x\in\mathbb R\setminus\mathbb Q$}\end{cases}$$
    is continuous at only one point. Is it differentiable at that point?
\end{ExerciseList}

\section{Primitives}
\begin{ExerciseList}
    \Exercise Calculate the primitive of the function $$\frac{2x+3}{x^2+2}$$ that vanishes at $x=0$

    \Exercise Determine a primitive for the following functions:
    \Question $\frac{1}{\sqrt{e^x - 1}}$
\end{ExerciseList}

\section{Integration}
\begin{ExerciseList}
    \Exercise Calculate the area of the following subset of $\mathbb R^2$
    $$\{(x,y)\in\mathbb R^2 \colon x\le y\le -x^2+2\}$$
\end{ExerciseList}

\section{Taylor Polynomial}
\begin{ExerciseList}
    \Exercise Let $f\in C^2(\mathbb R)$ and $g(x)=f(e^x)\forall x\in\mathbb R$.
    Let $3-x+2(x-1)^2$ be the second order Taylor polynomial of $f$ relative
    to point $1$, determine the second order MacLaurin polynomial of $g$.
\end{ExerciseList}
\section{Series}
\begin{ExerciseList}
    \Exercise Analyse the following series and determine whether they are absolutely convergent, conditionally convergent, or divergent
    \Question $\sum_{n=1}^{+\infty}\frac{1}{3+2n}$
    \Question $\sum_{n=1}^{+\infty}\frac{\cos(2n)}{n^3}$
    \Question $\sum_{n=1}^{+\infty}\frac{n^3}{e^{2n}}$
    \Exercise Show that if the series $\sum_{n=1}^{+\infty}a_n$ converges, then
    $\sum_{n=1}^{+\infty}\frac{2n-1}{2n}$ also converges.
\end{ExerciseList}

\section{Proofs}
\begin{ExerciseList}
    \Exercise Show that for any $x>0$
    $$\frac{x}{1+x} < \log(1+x) < x$$
    Hint: Mean Value Theorem

    \Exercise Let $h\in C(\mathbb R)$ such that $h(x) = h(x+2)$ for all $x\in\mathbb R$, and
    $$\phi(x) = \int_0^x{h(t)dt}-\int_0^{x+2}{h(t)dt}$$
    Prove that $\phi$ is identically null if and only if $\int_0^2 h(t)dt = 0$
\end{ExerciseList}
\section{Notation}
\begin{enumerate}
    \item  $[a,b]$ for a closed interval, $(a,b)$ for an open one.
    \item $\overline{\mathbb R} = \mathbb R \cup \{-\infty, +\infty\}$
\end{enumerate}
\end{document}
